\selectlanguage{english}
\clearpage

\section{Introduction}

This chapter introduces the overall topic and scope of the thesis. Section 1.1 outlines the motivation for developing a structured testing approach for IPC middleware. Section 1.2 defines the main objectives of the thesis, and Section 1.3 provides an overview of its structure.

\subsection{Motivation}

Inter-process communication (IPC) middleware plays a critical role in modern distributed systems. It enables software components (often running on different machines) to exchange data efficiently, coordinate actions, and maintain system coherence. This is especially important in domains such as robotics, automotive systems, and the Industrial Internet of Things (IIoT), where real-time requirements and fault tolerance are essential \cite{coulouris2012}.

\vspace{1em}
Despite the increasing reliance on IPC middleware, systematic and reusable testing concepts for these technologies are still lacking in many projects. While unit tests verify individual components in isolation, they are not sufficient for ensuring the correctness of interactions between distributed nodes, especially in failure-prone or time-sensitive environments.

\vspace{1em}
Middleware frameworks such as ROS 2 and DDS have introduced their own tools and mechanisms to support system-level validation e.g., in ROS 2  \texttt{launch\_testing} \cite{ros2_launch_testing}, or monitoring utilities in DDS implementations like eProsima Fast DDS and RTI Connext DDS \cite{eprosima_fast_dds, rti_connext_dds}. However, these approaches are often tightly coupled to the specifics of the respective middleware architectures and cannot be directly transferred to other systems.

\vspace{1em}
The enhanced Communication Abstraction Layer (eCAL) is an IPC framework designed for fast, scalable communication across processes and hosts \cite{ecal_github}. Although it provides a flexible communication model, eCAL currently lacks a standardized strategy for testing at the system level. While some components are covered by unit or functional tests, there is no unified infrastructure for evaluating end-to-end communication, message integrity under stress, or failure handling in realistic scenarios.

\vspace{1em}
This thesis addresses that gap by developing a generic testing concept for IPC middleware and applying it concretely to the case of eCAL.

\subsection{Objective}

The goal of this thesis is to develop and evaluate a testing concept that enables structured and automated validation of inter-process communication (IPC) middleware. A particular focus lies on the design of a reusable system-level testing methodology that can be adapted to different middleware architectures. As a practical example, the concept will be implemented and demonstrated using the eCAL framework.

\vspace{1em}
The core objectives are:

\begin{itemize}
	\item Design a structured testing approach for conducting system tests of IPC middleware in distributed environments.
	\item Investigate suitable tools, frameworks, and technologies for implementing automated tests.
	\item Apply the developed testing concept to eCAL and validate its applicability through representative test scenarios.
\end{itemize}

\subsection{Outline}

The structure of this thesis is as follows:

\begin{itemize}
	\item \textbf{Chapter 2: Theoretical Foundations}: Introduces testing principles and discusses core concepts of inter-process communication and middleware.
	\item \textbf{Chapter 3: Test Requirements for IPC Middleware}: Identifies key requirements for a testing strategy in IPC systems.
	\item \textbf{Chapter 4: Tool Evaluation and Selection}: Evaluates potential tools and frameworks for implementing the test system.
	\item \textbf{Chapter 5: Design and Automation of Tests}: Describes the test architecture and implementation of selected test cases.
	\item \textbf{Chapter 6: Conclusion and Outlook}: Summarizes the findings.
\end{itemize}







