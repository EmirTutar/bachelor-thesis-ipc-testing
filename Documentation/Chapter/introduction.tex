\selectlanguage{english}
\clearpage

\section{Introduction}

In this chapter, the topic of this thesis is introduced. Section 1.1 explains the motivation behind developing a systematic testing framework for eCAL-based inter-process communication. Then, section 1.2 describes the main objectives and goals of this thesis. Finally, section 1.3 provides an overview of the structure of the entire document.


\subsection{Motivation}

In modern software engineering, distributed systems have become very important due to their capability to handle large amounts of data efficiently and reliably. Middleware solutions, like the \textit{enhanced Communication Abstraction Layer} (eCAL), play an important role because they enable different software processes to exchange data and communicate with each other across multiple computing nodes \cite{ecal_github}. 
\\
\\
Reliability and correctness of these middleware solutions are especially important in areas such as automotive, robotics, and the Industrial Internet of Things (IIoT). Failures in communication could cause system breakdowns and significant safety risks, particularly in applications where real-time processing is essential \cite{middleware_challenges}. Therefore, it is critical to develop comprehensive testing strategies to ensure middleware solutions such as eCAL operate correctly and safely.
\\
\\
Currently, eCAL does not have a standardized approach for system-wide testing. Although individual parts of eCAL are tested through unit tests, these tests do not fully cover complex system scenarios and real communication patterns. Other popular middleware technologies, such as the Robot Operating System (ROS) or Data Distribution Service (DDS), have already established testing frameworks. However, these existing methods cannot be directly applied to eCAL because of differences in their architecture and design \cite{ipc_performance_analysis}. 
\\
\\
The creation of a test framework specifically for eCAL is therefore essential. Such a framework can significantly improve the quality and reliability of applications that depend on eCAL, especially for safety-critical use cases.

\clearpage

\subsection{Objective}

This thesis aims to develop and evaluate a dedicated system testing framework for eCAL-based IPC systems. The primary objectives are:

\begin{itemize}
	\item Design a structured approach for conducting system tests specific to eCAL.
	\item Implement the testing framework and assess its effectiveness in real-world scenarios.
	\item Explore integration possibilities with continuous integration and continuous deployment (CI/CD) pipelines to facilitate automated testing and early fault detection.
\end{itemize}

\subsection{Outline}

The structure of this thesis is as follows:

\begin{itemize}
	\item \textbf{Chapter 2: Theoretical Foundations} – Provides an overview of IPC principles, details the eCAL framework, and reviews existing testing methodologies.
	\item \textbf{Chapter 3: Framework Design} – Discusses the requirements and design considerations for the proposed testing framework.
	\item \textbf{Chapter 4: Implementation} – Details the development of test cases, simulation environments, and strategies for testing common failure scenarios.
	\item \textbf{Chapter 5: CI/CD Integration} – Explores the integration of the testing framework into CI/CD pipelines to enable automated testing.
	\item \textbf{Chapter 6: Evaluation} – Analyzes the framework's performance, including test coverage and execution efficiency.
	\item \textbf{Chapter 7: Conclusion and Future Work} – Summarizes the findings and suggests directions for future research.
\end{itemize}






